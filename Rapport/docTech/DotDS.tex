Le format Dump Set (.ds) comprte une ligne par dump.
Ces lignes sont écrites comme suit :

\begin{itemize}
        \item Le chemin du dump, relatif à l'empalcement du Dump Set
        \item Le format du dump, parmi les suivants :
        \begin{itemize}
                \item Binary, pour un fichier composé de '0' et de  '1' en ASCII
                \item Hexadecimal, pour un fichier écrit en hexadeximal avec des caractères ASCII
                \item Raw data, pour que le fichier soit lu directement
        \end{itemize}
\end{itemize}

Ces deux champs sont séparés par un point virgule.
Voici un exemple de fichier Dump Set : 

\begin{figure}[!h]
  \begin{center}
  RawData/TestRaw.txt;Raw data


  TestBin.txt;Binary


  ../TestHex.txt;Hexadecimal
  \end{center}
  \caption{MyDumpSet.ds}
  \label{dump_set_fromat}
\end{figure}
