Cette fonction, accessible via le menu Algorithms, permet mettre en évidence les chaînes de bits semblables au même endroit dans un dump.

Dans le cas de deux dumps, les similarités sont représentées par la couleur verte, tandis que les similarités sont représentées par la couleur rouge. Un exemple de similarités est représenté sur la figure \ref{04-1-sim_simple}. Pour faciliter la compréhension, on y a remplacé les bits (de sens à priori inconnu) par des lettres.

\begin{figure}[!h]
  \begin{center}
  {\tt\center
  {Dump 1 : \color{simColor} Ce}{\color{dissimColor} ci es}{\color{simColor}t }{\color{dissimColor} un exemple de }{\color{simColor} similarité}

  {Dump 2 : \color{simColor} Ce}{\color{dissimColor} la me}{\color{simColor}t }{\color{dissimColor} en couleur la }{\color{simColor} similarité}
  }
  \end{center}
  \caption{Exemple de similarités}
  \label{04-1-sim_simple}
\end{figure}

Afin d'affiner la recherche, il est possible de spécifier une taille de chaîne minimum, comme l'illustre la figure \ref{04-1-sim_taille_min}.

\begin{figure}[!h]
  \begin{center}
  {\tt
  {Dump 1 : \color{dissimColor} Ceci est un exemple de }{\color{simColor} similarité }{\color{dissimColor} avec une taille minimum de 4}

  {Dump 2 : \color{dissimColor} Cela met en couleur la }{\color{simColor} similarité }{\color{dissimColor} faisant plus de 4 caractères}
  }
  \end{center}
  \caption{Similarités avec une taille de chaîne minimum}
  \label{04-1-sim_taille_min}
\end{figure}


Dans le cas de plusieurs dumps, on dispose de trois couleurs. Le rouge représente les dissimilarités, le vert les similarités concernant le dump visualisé (c'est-à-dire les similarités commune à ce dump et à d'autres), tandis que le bleu correspond aux similarités ne concernant pas le dump visualisé (c'est-à-dire les similarités communes à d'autres dumps).
Ces couleurs ont des nuances : plus le vert ou le bleu sont prononcés, plus il y a de dumps partageant la similarité en question. La figure \ref{04-1-sim_mult} est un exemple de similarités avec 3 dumps.

\begin{figure}[!h]
  \begin{center}
  {\tt
  {Dump 1 : \color{dissimColor} Encore un}{\color{otherSimColor} e au}{\color{dissimColor} tre }{\color{simColor} similarité}{\color{dissimColor} .}

  {Dump 2 : \color{dissimColor} Toujours }{\color{simColor} plus }{\color{dissimColor} de }{\color{simColor} similarité}{\color{dissimColor} s}

  {Dump 3 : \color{dissimColor} colorées }{\color{simColor} plus }{\color{dissimColor} qu'}{\color{otherSimColor} auparavant}{\color{dissimColor} .}
  }
  \end{center}
  \caption{Similatités entre trois dumps}
  \label{04-1-sim_mult}
\end{figure}