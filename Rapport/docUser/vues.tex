Les vues sont accessibles via les différents onglets. Il y a pour le moment 3 vues :
\begin{description}
  \item[Text] \hfill \\
  Elle permet de visualiser l'intégralitée du dump sous l'encodage spécifié dans le selecteur "Encoding". Le décalage peut être géré via la case Offset. L'utilisation du bouton Add et de la case Label Name sera décrite dans la partie Labels.
  
  \item[Hexadecimal] \hfill \\
  Elle permet de visualiser le dump sous forme hexadécimale. Le nombre de bits est visible sur la gauche. Cette vue permet entre autre de repérer les motifs qui se répètent, tels que les séparateurs.

  \item[Bitmap] \hfill \\
  Elle permet de visualiser le dump courant sous forme de carrés de couleur. Cette vue permet entre autre de repérer les morceaux qui se répètent.
\end{description}

Chaque vue affiche le dump courant, c'est à dire le dump en surbrillance dans la zone Dumps Sets. Il est possible de changer de vue en cliquant sur l'onglet voulu.

