Avant toute chose, il est nécessaire de créer ou d'ouvrir un set de dumps; un set de dumps correspond à un groupe de dumps qui seront ouvert en même temps. Nous appellerons "Set" les sets de dumps.
Un set peut donc servir soit de fichier "projet", pour réouvrir tout les dumps précédement utilisés, ou de groupe de dumps similaires.
Un set est un fichier avec l'extention .ds qui contient les adresses sur le disque des dumps à ouvrir; si vous avez une erreur au chargement d'un set, vérifiez si les dumps sont aux bons endroits.
Dans le cas de la création d'un set, il faudra le nommer (en cliquant dessus) puis lui rajouter des dumps. Cela s'effectue en cliquant sur "Dumps/Add Dump".
Il peut arriver qu'un fichier se retrouve par erreur dans un set, ou qu'il ne soit plus nécessaire; la fonction Remove Dump du menu Dump permet de retirer du set le dump sélectionné.
Les fonctions "Save set" et "Save set as ..." du menu Dump permettent de sauvegarder un set, pour pouvoir reprendre le travail plus tard sur les mêmes données. Un set ne contients que les positions des Dumps, veillez à ne pas déplacer vos dumps ou les supprimer, ou vous risqueriez de provoquer des erreurs !
Un click sur un autre dump actualise l'interface avec ces nouvelles données.