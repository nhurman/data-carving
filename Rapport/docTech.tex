%%%%%%%%%%%%%%%%%%%%%%%%%%%%%%%%%%%%%%%%%%%%%%%%%%%%%%%%%%%%%%%%%%%%%%%%%%%%%%%
%     STYLE POUR LES EXPOSÉS TECHNIQUES
%         3e année INSA de Rennes
%
%             NE PAS MODIFIER
%%%%%%%%%%%%%%%%%%%%%%%%%%%%%%%%%%%%%%%%%%%%%%%%%%%%%%%%%%%%%%%%%%%%%%%%%%%%%%%

\documentclass[a4paper,11pt]{article}

\usepackage{exptech}       % Fichier (./exptech.sty) contenant les styles pour
                           % l'expose technique (ne pas le modifier)


%\linespread{1,6}          % Pour une version destinée à un relecteur,
                           % décommenter cette commande (double interligne)

% UTILISEZ SPELL (correcteur orthographique) à accès simplifié depuis XEmacs


%\setlength{\parskip}{2ex}
\usepackage{color}
\usepackage{graphicx}
\definecolor{simColor}{rgb}{0.0, 0.5, 0.0}
\definecolor{dissimColor}{rgb}{0.8, 0.0, 0.0}
\definecolor{otherSimColor}{rgb}{0.0, 0.28, 0.67}

%%%%%%%%%%%%%%%%%%%%%%%%%%%%%%%%%%%%%%%%%%%%%%%%%%%%%%%%%%%%%%%%%%%%%%%%%%%%%%%

\title{ \textbf{Data Carving  : Documentation Technique} }
\markright{\'Etude pratique - Data carving}
                           % Pour avoir le titre de l'expose sur chaque page

\author{Alexandre \textsc{Audinot}, Thierry \textsc{Gaugry}, \\
        Nicolas \textsc{Hurman}, Gabriel \textsc{Prevosto} \\
        \\
        Encadrant : Gildas \textsc{Avoine}}

\date{}                    % Ne pas modifier

%%%%%%%%%%%%%%%%%%%%%%%%%%%%%%%%%%%%%%%%%%%%%%%%%%%%%%%%%%%%%%%%%%%%%%%%%%%%%%%

\begin{document}

\maketitle                 % Génère le titre
\thispagestyle{empty}      % Supprime le numéro de page sur la 1re page



\begin{abstract}
Ceci est la doc technique
\end{abstract}

\section{Le Format DumpSet} Le format Dump Set (.ds) comprte une ligne par dump.
Ces lignes sont écrites comme suit :

\begin{itemize}
        \item Le chemin du dump, relatif à l'empalcement du Dump Set
        \item Le format du dump, parmi les suivants :
        \begin{itemize}
                \item Binary, pour un fichier composé de '0' et de  '1' en ASCII
                \item Hexadecimal, pour un fichier écrit en hexadeximal avec des caractères ASCII
                \item Raw data, pour que le fichier soit lu directement
        \end{itemize}
\end{itemize}

Ces deux champs sont séparés par un point virgule.
Voici un exemple de fichier Dump Set : 

\begin{figure}[!h]
  \begin{center}
  RawData/TestRaw.txt;Raw data


  TestBin.txt;Binary


  ../TestHex.txt;Hexadecimal
  \end{center}
  \caption{MyDumpSet.ds}
  \label{dump_set_fromat}
\end{figure}

\section{Les classes}
  \subsection{Similarities} \input{docTech/Similarities.tex}

\bibliography{biblio}

\end{document}
