L'étude bibliographique que nous avons réalisée ne nous a fourni que peu de pistes vers des travaux existants sur le sujet. En effet, il existe une multitude d'outils de \texttt{data carving}, mais ils sont le plus souvent destinés à de beaucoup plus vastes volumes de données, et en connaissant relativement précisément l'information recherchée (des fichers JPEG sur un disque dur abîmé par exemple). La particularité de notre sujet tient dans le fait que les volumes considérés sont extrêmement faibles, avec une structure qui varie énormément d'un support à l'autre. En effet, étant donné l'espace de stockage disponible sur une \texttt{smartcard}, les entreprises utilisent le plus souvent des structures qui leur sont exclusives.

Nous avons toutefois pu identifier un outil, mCarve \cite{mCarve}, qui illustrait un thème plus général mais correspondait en quelques points à ce que nous recherchions. Le défaut principal de mCarve est qu'il est essentiellement basé sur une analyse dynamique de dumps, où l'on possède plusieurs versions du même fichier capturées à des instants différents. Notre sujet portant également sur des analyses statiques, il fallait y ajouter un bon nombre de fonctionnalités. De plus, nous l'avons trouvé relativement peu intuitif à l'usage, l'interface utilisateur ne présentant que les résultats bruts des algorithmes de comparaison.
