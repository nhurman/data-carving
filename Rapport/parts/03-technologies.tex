Pour réaliser ce projet, nous avons utilisé un langage et des librairies multi-plateformes comme imposé par le cahier des charges. Le logiciel est écrit en C++, avec la librairie Qt \cite{Qt} pour l'interface graphique.

Un framework de tests unitaires, UnitTest++ \cite{UnitTest}, a également été mis en place pour tester les composants clé du logiciel. Doxygen \cite{Doxygen} nous a permis de générer la documentation technique.

Nos recherches ne nous ont ni permis de trouver des librairies de manipulation de chaînes de bits répondant à nos attentes, ni des implémentations des algorithmes utilisés satisfaisantes. Nous avons donc dû en écrire nos propres versions.
