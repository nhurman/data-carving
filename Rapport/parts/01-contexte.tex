Dans le cadre du projet \texttt{Études Pratiques} de 3ème année du département informatique, nous avons travaillé sur le sujet « Memory Carving Applied to SmardCard Dumps », avec l'aide de Gildas Avoine qui nous a accompagné tout au long de ce projet. Ce domaine nous étant totalement étranger, nous avons commencé par le découvrir à travers une recherche bibliographique. En partant du \texttt{dump} de la mémoire d'une smartcard, il est possible de retrouver les informations qui y sont contenues, ainsi que la manière dont elles sont structurées.

Les données peuvent être enregistrées de diverses manières suivant leur type. Pour le texte, il existe une multitude d'encodages tels que ASCII ou UTF-8 pour citer les plus connus. Les dates peuvent être sous la forme d'un nombre de secondes écoulées, au format JJ/MM/AAAA et bien d'autres. Devant le nombre de représentations différentes de la même donnée, réaliser un catalogue exhaustif des encodages serait une tâche immense et jamais terminée. Aussi nous nous sommes plus attardés sur une analyse des données permettant de mettre en avant leur structure, ou profitant des informations déjà connues (comme le nom de la personne) pour retrouver l'encodage utilisé et la localisation dans la mémoire du champ donné.
