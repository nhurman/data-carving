Nous avons réalisé notre logiciel pour répondre à la problématique qui nous était posée, et n'avons pas eu le temps d'implémenter d'autres fonctionnalités ; certaines nous sembleraient cependant réellement utiles.

\begin{description}
  \item[Optimisations] \hfill \\
  Nous n'avons pas réalisé de tests de performances avec des outils de profiling.
  Il serait judicieux de le faire, en particulier sur les algorithmes, ce qui pourrait permettre de diviser le temps nécessaire à l'analyse des dumps. En effet, certains de nos algorithmes ont une complexité qui n'est peut-être pas la meilleure.

  \item[Masques préchargés] \hfill \\
  Il serait utile de réaliser un ensemble de masques livrés avec le logiciel permettant d'ouvrir des dumps provenant des cartes les plus répandues sans avoir à refaire le travail d'analyse.

  \item[Environnement plus intégré] \hfill \\
  Actuellement, plusieurs outils sont nécessaires : un pour l'extraction du dump depuis la carte, un second pour analyser le dump et éventuellement un troisième pour mettre en forme les résultats. Intégrer les fonctionnalités de notre logiciel dans un environnement permettant de réaliser l'ensemble de la procédure à travers la même interface apporterait une meilleure expérience utilisateur.

  \item[Davantage de visualisations] \hfill \\
  Certains types de données, comme les images, ne sont pas prises en charge par notre logiciel. Pour certaines mémoires, comme celles des passeports, ce type de visualisation serait intéressant.

  \item[Plugins pour les encodages] \hfill \\
  Nous avons volontairement limité le nombre d'encodages pris en charge, pour ne pas perdre de temps et pouvoir nous concentrer sur des fonctionnalités plus intéressantes. Pouvoir exporter les encodages dans des fichiers séparés permettrait à chaque utilisateur de définir son encodage s'il n'existe pas déjà et de les partager.

  \item[Traduire le logiciel] \hfill \\
  L'interface a été réalisée en anglais, une internationalisation la rendrait plus conviviale pour les différents utilisateurs.
\end{description}
