Lors de l’étude préliminaire du projet, nous avions établi un cahier des charges. Voici un tableau récapitulatif du travail accompli par rapport au résultat attendu.


\begin{tabular}{|p{6cm}|p{6cm}|}
  \hline
  Permettre à l’utilisateur d’extraire des données d’un petit dump de mémoire de structure inconnue &
  L'utilisateur peut trouver des informations en appliquant différents encodages à un dump quelque soit sa structure
  \\ \hline
  Être utilisable sur un dump unique ou sur un ensemble de dumps &
  L'application permet de créer des DumpSets qui regroupent un ou plusieurs dumps
  \\ \hline
  Donne la possibilité à l’utilisateur de garder une trace des données déjà extraites &
  
  \\ \hline
  Peut traduire le binaire vers différents encodages et vice versa &
  L'utilisateur peut choisir comment afficher le dump courant parmi une liste d'encodages
  \\ \hline
  Fonctionne sous windows et linux &
  Le programme a été testé et fonctionne sous les deux environnements
  \\ \hline \hline
  \textbf{Optionnel} &
  \\ \hline
  Donne une représentation visuelle des données grâce à un code couleur &
  
  \\ \hline
  Permet à l’utilisateur de créer des masques et de les réutiliser &
  
  \\ \hline
  Peut reconnaître les images dans les dumps &
  Objectif non achevé par manque de temps
  \\ \hline
\end{tabular}
