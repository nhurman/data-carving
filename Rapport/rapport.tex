%%%%%%%%%%%%%%%%%%%%%%%%%%%%%%%%%%%%%%%%%%%%%%%%%%%%%%%%%%%%%%%%%%%%%%%%%%%%%%%
%     STYLE POUR LES EXPOSÉS TECHNIQUES
%         3e année INSA de Rennes
%
%             NE PAS MODIFIER
%%%%%%%%%%%%%%%%%%%%%%%%%%%%%%%%%%%%%%%%%%%%%%%%%%%%%%%%%%%%%%%%%%%%%%%%%%%%%%%

\documentclass[a4paper,11pt]{article}

\usepackage{exptech}       % Fichier (./exptech.sty) contenant les styles pour
                           % l'expose technique (ne pas le modifier)

\setlength{\parskip}{2ex}

%\linespread{1,6}          % Pour une version destinée à un relecteur,
                           % décommenter cette commande (double interligne)

% UTILISEZ SPELL (correcteur orthographique) à accès simplifié depuis XEmacs

%%%%%%%%%%%%%%%%%%%%%%%%%%%%%%%%%%%%%%%%%%%%%%%%%%%%%%%%%%%%%%%%%%%%%%%%%%%%%%%

\title{ \textbf{Data Carving applied to small memory dumps} }
\markright{\'Etude pratique - Data carving}
                           % Pour avoir le titre de l'expose sur chaque page

\author{Alexandre \textsc{Audinot}, Thierry \textsc{Gaugry}, \\
        Nicolas \textsc{Hurman}, Gabriel \textsc{Prevosto} \\
        \\
        Encadrant : Gildas \textsc{Avoine}}

\date{}                    % Ne pas modifier

%%%%%%%%%%%%%%%%%%%%%%%%%%%%%%%%%%%%%%%%%%%%%%%%%%%%%%%%%%%%%%%%%%%%%%%%%%%%%%%

\begin{document}

\maketitle                 % Génère le titre
\thispagestyle{empty}      % Supprime le numéro de page sur la 1re page



\begin{abstract}
Nous possédons une multitude d'appareils que nous utilisons chaque jour, parfois à notre insu. Quelles informations enregistrent-ils ?
Au cours de cette étude pratique, nous avons essayé de développer un logiciel permettant de comprendre la structure d'une mémoire de petite taille, comme on peut en trouver dans des cartes de transport, des pass de ski ou encore dans l'électronique embarquée de nos véhicules. En appliquant une série d'algorithmes et à travers une interface intuitive, disséquer ce type de support devient une tâche plus simple et accessible.
\end{abstract}

\section{Introduction}
  \subsection{Contexte}
  \subsection{Etude de l'existant}
\section{Cahier des charges}
  \subsection{Spécifications originales}
  \subsection{État final}
\section{Présentation du logiciel}
  \subsection{Technologies utilisées}
  \subsection{Interface}
  \subsection{Fonctionnalités}
\section{Algorithmique}
  \subsection{Aide à la décision}
  \subsection{Analyse}
\section{Implémentation}
  \subsection{Pseudo-code}
  \subsection{Complexité}
  \subsection{Performances}
\section{Gestion du projet}
  \subsection{Outils}
  \subsection{Planification et réunions}
  \subsection{Versions intermédiaires}
  \subsection{Séparation du travail}
  \subsection{Distribution du temps}
\section{Conclusion}
  \subsection{Points remarquables au cours du projet}
  \subsection{Améliorations envisageables}
    1. Optimisations
    2. Masques préchargés
    3. Environnement plus intégré
    4. Davantage de visualisations
    5. Plugins pour les encodages
    6. Traduire le logiciel
\bibliography{biblio}

\end{document}
